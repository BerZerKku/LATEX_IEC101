% при использовании \setcounter{..} происходит сдвиг правой границы таблицы
% для того чтобы избежать этого, "\\ \hline" надо писать после него без пробелов
\newcounter{adr}
\newcommand{\cntadr}{%
\arabic{adr}\stepcounter{adr}%
}

\begin{tabularx}{\linewidth}{| *{3}{M{1cm} |} *{5}{M{0.3cm} |} X|}  
 	\caption{Карта памяти}	\label{tab:appMap} 
	\tabularnewline
    
    \firsthline
    \multicolumn{3}{|c|}{Адрес объекта}							& 
    \multicolumn{5}{c|}{Совместимость} 							& 
    \centering \multirow{2}{*}{Описание} \tabularnewline \cline{1-8}
    \begin{sideways} МЭК~101~~~~~~~~~~~~~~~~ \end{sideways} 	& 
    \begin{sideways} МЭК~104 (Аппарат 1) \end{sideways}			&
    \begin{sideways} МЭК~104 (Аппарат 2) \end{sideways}			&
    \begin{sideways} Р400~~~~~~~~~~~~~~~~~~~~~~ \end{sideways}	& 
    \begin{sideways} РЗСК~~~~~~~~~~~~~~~~~~~~~ \end{sideways}	&
    \begin{sideways} К400~ВЧ~~~~~~~~~~~~~~~~~~\end{sideways}	& 
    \begin{sideways} К400 ОПТИКА~~~~~~~~~ \end{sideways} 		& 
    \begin{sideways} К400 КОЛЬЦО~~~~~~~~~ \end{sideways} 		& 
   	\tabularnewline \hline
    \endfirsthead

	\multicolumn{9}{l}{Продолжение таблицы~\ref{tab:appMap}}
	\tabularnewline \hline
   	\multicolumn{3}{|c|}{Адрес объекта}							&
    \multicolumn{5}{c|}{Совместимость} 							&
    \centering \multirow{2}{*}{Описание} \tabularnewline \cline{1-8}
    \begin{sideways} МЭК~101~~~~~~~~~~~~~~~~ \end{sideways} 	&
    \begin{sideways} МЭК~104 (Аппарат 1) \end{sideways}			&
    \begin{sideways} МЭК~104 (Аппарат 2) \end{sideways}			&
    \begin{sideways} Р400~~~~~~~~~~~~~~~~~~~~~~ \end{sideways}	&
    \begin{sideways} РЗСК~~~~~~~~~~~~~~~~~~~~~ \end{sideways}	&
    \begin{sideways} К400~ВЧ~~~~~~~~~~~~~~~~~~\end{sideways}	&
    \begin{sideways} К400 ОПТИКА~~~~~~~~~ \end{sideways} 		&
    \begin{sideways} К400 КОЛЬЦО~~~~~~~~~ \end{sideways} 		&
   	\tabularnewline \hline
  	\endhead

    \multicolumn{9}{r}{продолжение следует\ldots}
	\endfoot
	\endlastfoot

%%%	
	\multicolumn{9}{|c|}{Дискретные входы TS32.} 
	\tabularnewline \hline

					& % Адрес 101
	2001 2002 ... 2032	& % Адрес 104 (аппарат 1)
					& % Адрес 104 (аппарат 2)
			 		& % Р400
	$\bullet$ 		& % РЗСК
	$\bullet$ 		& % К400 ВЧ
	$\bullet$ 		& % К400 ОПТИКА
	$\bullet$		& % К400 КОЛЬЦО
	Дискретный вход (команда) 01-32.
	\tabularnewline \hline
	
					& % Адрес 101
	2034			& % Адрес 104 (аппарат 1)
					& % Адрес 104 (аппарат 2)
			 		& % Р400
	$\bullet$ 		& % РЗСК
	$\bullet$ 		& % К400 ВЧ
	$\bullet$ 		& % К400 ОПТИКА
	$\bullet$		& % К400 КОЛЬЦО
	TS32 неисправен.
	\tabularnewline \hline
	
%%%	
	\multicolumn{9}{|c|}{Дискретные каналы MT500.} 
	\tabularnewline \hline
	
					& % Адрес 101
	2101			& % Адрес 104 (аппарат 1)
					& % Адрес 104 (аппарат 2)
			 		& % Р400
	$\bullet$ 		& % РЗСК
	$\bullet$ 		& % К400 ВЧ
	$\bullet$ 		& % К400 ОПТИКА
	$\bullet$		& % К400 КОЛЬЦО
	Вывод команд ПРМ (контроль цепи).
	\tabularnewline \hline
	
					& % Адрес 101
	2102			& % Адрес 104 (аппарат 1)
					& % Адрес 104 (аппарат 2)
			 		& % Р400
	$\bullet$ 		& % РЗСК
	$\bullet$ 		& % К400 ВЧ
	$\bullet$ 		& % К400 ОПТИКА
	$\bullet$		& % К400 КОЛЬЦО
	Работа ПРД.
	\tabularnewline \hline
	
					& % Адрес 101
	2103			& % Адрес 104 (аппарат 1)
					& % Адрес 104 (аппарат 2)
			 		& % Р400
	$\bullet$ 		& % РЗСК
	$\bullet$ 		& % К400 ВЧ
	$\bullet$ 		& % К400 ОПТИКА
	$\bullet$		& % К400 КОЛЬЦО
	Работа ПРМ.
	\tabularnewline \hline
	
					& % Адрес 101
	2104			& % Адрес 104 (аппарат 1)
					& % Адрес 104 (аппарат 2)
			 		& % Р400
	$\bullet$ 		& % РЗСК
	$\bullet$ 		& % К400 ВЧ
	$\bullet$ 		& % К400 ОПТИКА
	$\bullet$		& % К400 КОЛЬЦО
	Авария.
	\tabularnewline \hline
	
					& % Адрес 101
	2105			& % Адрес 104 (аппарат 1)
					& % Адрес 104 (аппарат 2)
			 		& % Р400
	$\bullet$ 		& % РЗСК
	$\bullet$ 		& % К400 ВЧ
	$\bullet$ 		& % К400 ОПТИКА
	$\bullet$		& % К400 КОЛЬЦО
	Предупреждение.
	\tabularnewline \hline
	
					& % Адрес 101
	2106			& % Адрес 104 (аппарат 1)
					& % Адрес 104 (аппарат 2)
			 		& % Р400
	$\bullet$ 		& % РЗСК
	$\bullet$ 		& % К400 ВЧ
	$\bullet$ 		& % К400 ОПТИКА
	$\bullet$		& % К400 КОЛЬЦО
	Контроль ключей ПРМ.
	\tabularnewline \hline
	
					& % Адрес 101
	2107			& % Адрес 104 (аппарат 1)
					& % Адрес 104 (аппарат 2)
			 		& % Р400
	$\bullet$ 		& % РЗСК
	$\bullet$ 		& % К400 ВЧ
	$\bullet$ 		& % К400 ОПТИКА
	$\bullet$		& % К400 КОЛЬЦО
	Вывод команд ПРД (контроль цепи).
	\tabularnewline \hline
	
					& % Адрес 101
	2108			& % Адрес 104 (аппарат 1)
					& % Адрес 104 (аппарат 2)
			 		& % Р400
	$\bullet$ 		& % РЗСК
	$\bullet$ 		& % К400 ВЧ
	$\bullet$ 		& % К400 ОПТИКА
	$\bullet$		& % К400 КОЛЬЦО
	Резерв.
	\tabularnewline \hline
	
%%%
 	\multicolumn{9}{|c|}{Информация о текущем состоянии (данные класса 2).} 
 	\tabularnewline \hline
	
	201			& % Адрес 101
	2201		& % Адрес 104 (аппарат 1)
	3201		& % Адрес 104 (аппарат 2)
	$\bullet$ 	& % Р400
	$\bullet$ 	& % РЗСК
	$\bullet$ 	& % К400 ВЧ
	$\bullet$ 	& % К400 ОПТИКА 
	$\bullet$ 	& % К400 КОЛЬЦО
	Индикация неисправности.	 							
	\tabularnewline \hline

	202			& % Адрес 101
	2202		& % Адрес 104 (аппарат 1)
	3202		& % Адрес 104 (аппарат 2)
	$\bullet$ 	& % Р400
	$\bullet$ 	& % РЗСК
	$\bullet$ 	& % К400 ВЧ
	$\bullet$ 	& % К400 ОПТИКА 
	$\bullet$ 	& % К400 КОЛЬЦО 
	Индикация предупреждения.	 
	\tabularnewline \hline
	
	203			& % Адрес 101
	2203		& % Адрес 104 (аппарат 1)
	3203		& % Адрес 104 (аппарат 2)
			 	& % Р400
	$\bullet$ 	& % РЗСК
	$\bullet$ 	& % К400 ВЧ
	$\bullet$ 	& % К400 ОПТИКА 
	$\bullet$ 	& % К400 КОЛЬЦО
	Индикация команд передатчика.	 
	\tabularnewline \hline
	
	204			& % Адрес 101
	2204		& % Адрес 104 (аппарат 1)
	3204		& % Адрес 104 (аппарат 2)
			 	& % Р400
	$\bullet$ 	& % РЗСК
	$\bullet$ 	& % К400 ВЧ
	$\bullet$ 	& % К400 ОПТИКА 
	$\bullet$ 	& % К400 КОЛЬЦО
	Индикация команд приемника.	
	\tabularnewline \hline
	
	301  302 ...  316	& % Адрес 101
	2301 2302 ... 2316	& % Адрес 104 (аппарат 1)
	3301 3302 ... 3316	& % Адрес 104 (аппарат 2)
	$\bullet$ 	& % Р400
	$\bullet$ 	& % РЗСК
	$\bullet$ 	& % К400 ВЧ
	$\bullet$ 	& % К400 ОПТИКА 
	$\bullet$ 	& % К400 КОЛЬЦО
	Индикация общей неисправности с кодом 0х0001, 0x0002 \ldots 0x8000.	 
	\tabularnewline \hline 
	 
	\lasthline
\end{tabularx}

Информация по неисправностям и предупрежденям приведена в Д-РО на соотвествующий аппарат. 




