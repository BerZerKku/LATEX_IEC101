\documentclass[russian,utf8,pointsection]{eskdtext}
\usepackage{cmap}			% Поиск по русским словам в конечном pdf документе
\usepackage{eskdchngsheet}
\usepackage[T2A]{fontenc}
\usepackage{pscyr}			% Подключение "красивых" шрифтов киррилицы
\usepackage{amstext}
\usepackage{amsmath}
\usepackage{listings}
\usepackage{color}
\usepackage{ifthen}

% работа с сылками
\usepackage{hyperref}
% \usepackage[usenames,dvipsnames,svgnames,table,rgb]{xcolor} 
\hypersetup{			
	unicode=true,           				% русские буквы в разделе PDF
	pdftitle={Аппаратура АВАНТ},			% Заголовок
	pdfauthor={Щеблыкин М.В.}, 				% Автор
	pdfsubject={Руководство по использованию протокола МЭК 870-5-101},	% Тема 
	colorlinks=true,      					% false: ссылки в рамках; true: цветные ссылки 
	linkcolor=black,        				% внутренние ссылки
	citecolor=blue,        					% на библиографию
	filecolor=blue,        					% на файлы
	urlcolor =blue          				% на URL 
}

% Дает доступ к командам:
% \MakeTextUppercase{} - сделать все символы заглавными
\usepackage{textcase} 

%Изменение отображения Содержания
%\makeatletter
%\renewcommand{\l@section}{\@dottedtocline{1}{0em}{1.25em}}
%\renewcommand{\l@subsection}{\@dottedtocline{2}{1.25em}{1.75em}}
%\renewcommand{\l@subsubsection}{\@dottedtocline{3}{2.75em}{2.6em}}
%\makeatother

% Работа с таблицами
% p{} - top align, m{} - middle align, b{} - bottom align
\usepackage{ltablex} 										% longtable с функциональностю tabularx
\usepackage{multirow} 										% Слияние строк в таблице
\renewcommand{\tabularxcolumn}[1]{>{\arraybackslash}m{#1}}	% выравнивание в ячейке таблицы по середине по вертикали
\newcolumntype{M}[1]{>{\centering \arraybackslash}m{#1}} 	% колонка с заданной шириной и выравниванием по центру
\newcolumntype{Z}{>{\centering \arraybackslash} X} 			% колонка с выравниванием по центру

% Заполнение граф Титульного листа и основной надписи
\ESKDcompany{ООО <<Прософт-Системы>>}
\ESKDclassCode{ОКП xx xxxx}
\ESKDtitle{Аппаратура <<АВАНТ>>}
\ESKDdocName{Руководство по использованию протокола МЭК 870-5-101}
\ESKDsignature{ПБКМ.424325.00x РЭ}
\ESKDgroup{\normalsize ООО <<Прософт-Системы>>}

% основная надпись
\ESKDauthor{\ESKDfontII Щеблыкин М.В.}
\ESKDchecker{\ESKDfontII Макаров Е.Г.}
%\ESKDnormContr{\ESKDfontII Бунина О.Ю.}
\ESKDapprovedBy{\ESKDfontII Чирков А.Г.}
\ESKDdate{2018/06/28}
\ESKDcolumnI{Аппаратура <<АВАНТ>> \\ \vspace{0.5cm} \ESKDfontIII{Руководство по использованию протокола МЭК870-5-101}}

% Добавлено отображение Города на Титульном листе
\renewcommand{\ESKDtheTitleFieldX}{Екатеринбург \\ \ESKDtheYear}

% Уменьшен размер шрифта для заголовков секций
\ESKDsectStyle{section}{\large \bfseries \MakeTextUppercase}

% Выравнивание по центру.
% Предназначено для выравнивания надписей в шапке таблицы
\newcommand{\calign}[1]{\centering #1 \arraybackslash} 

\begin{document} 
	\maketitle
	\tableofcontents
	
	\newpage

	%%% ----------
\section{Общие сведения} \label{sec:overview}

К информационно-управляющей сети аппаратура <<АВАНТ>> может быть подключена с~помощью коммуникационного прота RS-485/RS-422 и~поддерживаемого протокола МЭК 870-5-101 (ГОСТ Р МЭК 870-5-101).  

Реализованный в аппаратуре <<АВАНТ>> протокол соотвествует стандартному протоколу МЭК 870-5-101.

В~сети МЭК 870-5-101 <<АВАНТ>> всегда выступает в~роли контролируемого пункта (КП) .

Реализованный протокол МЭК 870-5-101 обеспечивает считывание флагов приемопередатчика, список и~описание которых приведен в~приложении~\ref{tab:appMap101}.

Функции протокола реализованы в~блоке БСП (плата БСП-ПИ). Выводы подключения находятся на~клеммнике КВП.
	%%% ----------
\section{Технические характеристики} \label{sec:tth}

\begin{list}{--}{Технические характеристики:}
	\item Физический уровень: последовательный канал стандарта RS-485/RS-422. Цепи подключения линии гальванически развязаны с~остальными цепями и~корпусом аппарата (уровень изоляции 2500~эфф.).
	\item Максимальная длина линии <<пункт управления - <<АВАНТ>>>> определеятеся типом кабеля и~скоростью передачи. Рекомендуется использовать экранированную витую пару.
	\item Скорости обмена данными: 600, 1200, 2400, 4800, 9600, 19200~бит/с.
	\item Адреса устройства в~сети МЭК 870-5-101: от~1 до~247.
\end{list}

При передаче данных исползуется формат кадра FT1.2, определенный в ГОСТ~Р~МЭК~870-5-2. Допускается формат как с~фиксированной, так и с~переменной длиной блока.

Аппаратура <<АВАНТ>> поддерживает только небалансную передачу по каналу.

Для передачи данных используется только режим <<1>> (младший байт передается первым). 
 
В~случае обнаружения ошибки прием кадра прекращается. Для возобновления процесса приема кадров требуется, чтобы приемник зафиксировал, что входная линия перешла в~исходное состояние (уровень сигнала в~линии соответсвует уровню стопового бита) и~удерживается в~этом состоянии в~течение времени передачи не менее 33~бит.

\begin{list}{--}{Поддерживаются следующие стандартные идентификаторы типа (ASDU):}
	\item M\_SP\_NA\_1 1 Одноэлементная информация без~метки времени.
	\item M\_SP\_TB\_1 30 Одноэлементная информация с~меткой времени \\ СР56Время2а.
	\item M\_EI\_NA\_1 70 Окончение инициализации.
	\item C\_IC\_NA\_1 100 Команда опроса.
	\item C\_CS\_NA\_1 103 Команда синхронизации часов.
\end{list}

\begin{list}{--}{Фиксированные параметры пакета МЭК~870-5-101:} 
	\item Общий адрес ASDU : 1.
	\item Длина адреса станции (байт): 1.
	\item Длина общего адреса ASDU (байт): 1.
	\item Длина адреса объекта информации (байт): 2.
	\item Длина кода причины передачи (байт): 1.
\end{list}

Процедура опроса обеспечивается на~канальном уровне, который запрашивает пользовательские данные классов~1 и 2. 

Данные класса~2 можно считать командой опроса, при этом они будут переданы без метки времени.
	%%% ----------
\section{Установка параметров соединения} \label{sec:setup}

Параметры соединения должны быть настроены до~установки связи. 
 
Настройка параметров соединения производится с~пульта управления блока БСП в~меню <<Настройка/Интерфейс>>. Параметры соединения хранятся в~ПЗУ и не~требуют повторной настройки при~следующем включении питания.

\textbf{Изменение параметров соединения с~пульта управления при~установленном соединении приводит к~потере связи.}

В~таблице~\ref{tab:connection} приведены используемые параметры и их рекомендованные значения.

\begin{tabularx}{\linewidth}{| M{3cm} | X | M{4cm} | M{2.5cm} |}
	\caption{Параметры соединения} \label{tab:connection} \\
    
    \hline
    Название параметра			& 
    \calign{Описание параметра}	& 
    Возможные значения 			& 
    Значение \\ \hline 
    \endfirsthead
    
    \multicolumn{4}{r}{продолжение следует\ldots} \\  
    \endfoot 
	\endlastfoot
	
	\hline
	Название параметра 			&		
	\calign{Описание параметра}	&														
	Возможные значения			&
	Значение 					\\ \hline 
	\endhead
	
	Интерфейс связи				& 
	Выбор внешнего интерфейса связи.	& 
	USB, RS485 					& 
	RS485						\\ \hline
	
	Протокол					& 
	Текущий протокол связи.	 	& 
	Стандарт, MODBUS, МЭК-101	& 
	МЭК-101						\\ \hline
	
	Сетевой адрес (link)		& 
	Адрес аппарата в локальной сети.	& 
	от~1 до~247 				& 
	1							\\ \hline
	
	Биты данных					& 
	Число информационных бит в передаваемых и принимаемых байтах. & 
	8 							& 
	8~бит						\\ \hline
	
	Скорость передачи			& 
	Скорость передачи данных.	& 
	600, 1200, 2400, 4800, 9600, 19200 бит/с	& 
	19200						\\ \hline
	
	Четность					& 
	Схема контроля четности.	& 
	нет, чет, нечет			 	& 
	чет							\\ \hline
	
	Стоповые биты				& 
	Количество стоповых бит.	& 
	1 или 2						& 
	1~бит						\\ \hline
\end{tabularx}

Выбор используемого интерфейса RS-485/RS-422 осуществляется переключателями на~блоке КВП. Если аппарат является крайним в~цепи линии связи, то~рекомендуется подключить согласующий резистор номиналом 120~Ом (переключатель на блоке КВП). 


	
 	\ESKDappendix{Обязательное}{Карта памяти} \label{app:map}
 	% при использовании \setcounter{..} происходит сдвиг правой границы таблицы
% для того чтобы избежать этого, "\\ \hline" надо писать после него без пробелов
\newcounter{adr}
\newcommand{\cntadr}{%
\arabic{adr}\stepcounter{adr}%
}

\begin{tabularx}{\linewidth}{| *{3}{M{1cm} |} *{5}{M{0.3cm} |} X|}  
 	\caption{Карта памяти}	\label{tab:appMap} 
	\tabularnewline
    
    \firsthline
    \multicolumn{3}{|c|}{Адрес объекта}							& 
    \multicolumn{5}{c|}{Совместимость} 							& 
    \centering \multirow{2}{*}{Описание} \tabularnewline \cline{1-8}
    \begin{sideways} МЭК~101~~~~~~~~~~~~~~~~ \end{sideways} 	& 
    \begin{sideways} МЭК~104 (Аппарат 1) \end{sideways}			&
    \begin{sideways} МЭК~104 (Аппарат 2) \end{sideways}			&
    \begin{sideways} Р400~~~~~~~~~~~~~~~~~~~~~~ \end{sideways}	& 
    \begin{sideways} РЗСК~~~~~~~~~~~~~~~~~~~~~ \end{sideways}	&
    \begin{sideways} К400~ВЧ~~~~~~~~~~~~~~~~~~\end{sideways}	& 
    \begin{sideways} К400 ОПТИКА~~~~~~~~~ \end{sideways} 		& 
    \begin{sideways} К400 КОЛЬЦО~~~~~~~~~ \end{sideways} 		& 
   	\tabularnewline \hline
    \endfirsthead

	\multicolumn{9}{l}{Продолжение таблицы~\ref{tab:appMap}}
	\tabularnewline \hline
   	\multicolumn{3}{|c|}{Адрес объекта}							&
    \multicolumn{5}{c|}{Совместимость} 							&
    \centering \multirow{2}{*}{Описание} \tabularnewline \cline{1-8}
    \begin{sideways} МЭК~101~~~~~~~~~~~~~~~~ \end{sideways} 	&
    \begin{sideways} МЭК~104 (Аппарат 1) \end{sideways}			&
    \begin{sideways} МЭК~104 (Аппарат 2) \end{sideways}			&
    \begin{sideways} Р400~~~~~~~~~~~~~~~~~~~~~~ \end{sideways}	&
    \begin{sideways} РЗСК~~~~~~~~~~~~~~~~~~~~~ \end{sideways}	&
    \begin{sideways} К400~ВЧ~~~~~~~~~~~~~~~~~~\end{sideways}	&
    \begin{sideways} К400 ОПТИКА~~~~~~~~~ \end{sideways} 		&
    \begin{sideways} К400 КОЛЬЦО~~~~~~~~~ \end{sideways} 		&
   	\tabularnewline \hline
  	\endhead

    \multicolumn{9}{r}{продолжение следует\ldots}
	\endfoot
	\endlastfoot

%%%	
	\multicolumn{9}{|c|}{Дискретные входы TS32.} 
	\tabularnewline \hline

					& % Адрес 101
	2001 2002 ... 2032	& % Адрес 104 (аппарат 1)
					& % Адрес 104 (аппарат 2)
			 		& % Р400
	$\bullet$ 		& % РЗСК
	$\bullet$ 		& % К400 ВЧ
	$\bullet$ 		& % К400 ОПТИКА
	$\bullet$		& % К400 КОЛЬЦО
	Дискретный вход (команда) 01-32.
	\tabularnewline \hline
	
					& % Адрес 101
	2034			& % Адрес 104 (аппарат 1)
					& % Адрес 104 (аппарат 2)
			 		& % Р400
	$\bullet$ 		& % РЗСК
	$\bullet$ 		& % К400 ВЧ
	$\bullet$ 		& % К400 ОПТИКА
	$\bullet$		& % К400 КОЛЬЦО
	TS32 неисправен.
	\tabularnewline \hline
	
%%%	
	\multicolumn{9}{|c|}{Дискретные каналы MT500.} 
	\tabularnewline \hline
	
					& % Адрес 101
	2101			& % Адрес 104 (аппарат 1)
					& % Адрес 104 (аппарат 2)
			 		& % Р400
	$\bullet$ 		& % РЗСК
	$\bullet$ 		& % К400 ВЧ
	$\bullet$ 		& % К400 ОПТИКА
	$\bullet$		& % К400 КОЛЬЦО
	Вывод команд ПРМ (контроль цепи).
	\tabularnewline \hline
	
					& % Адрес 101
	2102			& % Адрес 104 (аппарат 1)
					& % Адрес 104 (аппарат 2)
			 		& % Р400
	$\bullet$ 		& % РЗСК
	$\bullet$ 		& % К400 ВЧ
	$\bullet$ 		& % К400 ОПТИКА
	$\bullet$		& % К400 КОЛЬЦО
	Работа ПРД.
	\tabularnewline \hline
	
					& % Адрес 101
	2103			& % Адрес 104 (аппарат 1)
					& % Адрес 104 (аппарат 2)
			 		& % Р400
	$\bullet$ 		& % РЗСК
	$\bullet$ 		& % К400 ВЧ
	$\bullet$ 		& % К400 ОПТИКА
	$\bullet$		& % К400 КОЛЬЦО
	Работа ПРМ.
	\tabularnewline \hline
	
					& % Адрес 101
	2104			& % Адрес 104 (аппарат 1)
					& % Адрес 104 (аппарат 2)
			 		& % Р400
	$\bullet$ 		& % РЗСК
	$\bullet$ 		& % К400 ВЧ
	$\bullet$ 		& % К400 ОПТИКА
	$\bullet$		& % К400 КОЛЬЦО
	Авария.
	\tabularnewline \hline
	
					& % Адрес 101
	2105			& % Адрес 104 (аппарат 1)
					& % Адрес 104 (аппарат 2)
			 		& % Р400
	$\bullet$ 		& % РЗСК
	$\bullet$ 		& % К400 ВЧ
	$\bullet$ 		& % К400 ОПТИКА
	$\bullet$		& % К400 КОЛЬЦО
	Предупреждение.
	\tabularnewline \hline
	
					& % Адрес 101
	2106			& % Адрес 104 (аппарат 1)
					& % Адрес 104 (аппарат 2)
			 		& % Р400
	$\bullet$ 		& % РЗСК
	$\bullet$ 		& % К400 ВЧ
	$\bullet$ 		& % К400 ОПТИКА
	$\bullet$		& % К400 КОЛЬЦО
	Контроль ключей ПРМ.
	\tabularnewline \hline
	
					& % Адрес 101
	2107			& % Адрес 104 (аппарат 1)
					& % Адрес 104 (аппарат 2)
			 		& % Р400
	$\bullet$ 		& % РЗСК
	$\bullet$ 		& % К400 ВЧ
	$\bullet$ 		& % К400 ОПТИКА
	$\bullet$		& % К400 КОЛЬЦО
	Вывод команд ПРД (контроль цепи).
	\tabularnewline \hline
	
					& % Адрес 101
	2108			& % Адрес 104 (аппарат 1)
					& % Адрес 104 (аппарат 2)
			 		& % Р400
	$\bullet$ 		& % РЗСК
	$\bullet$ 		& % К400 ВЧ
	$\bullet$ 		& % К400 ОПТИКА
	$\bullet$		& % К400 КОЛЬЦО
	Резерв.
	\tabularnewline \hline
	
%%%
 	\multicolumn{9}{|c|}{Информация о текущем состоянии (данные класса 2).} 
 	\tabularnewline \hline
	
	201			& % Адрес 101
	2201		& % Адрес 104 (аппарат 1)
	3201		& % Адрес 104 (аппарат 2)
	$\bullet$ 	& % Р400
	$\bullet$ 	& % РЗСК
	$\bullet$ 	& % К400 ВЧ
	$\bullet$ 	& % К400 ОПТИКА 
	$\bullet$ 	& % К400 КОЛЬЦО
	Индикация неисправности.	 							
	\tabularnewline \hline

	202			& % Адрес 101
	2202		& % Адрес 104 (аппарат 1)
	3202		& % Адрес 104 (аппарат 2)
	$\bullet$ 	& % Р400
	$\bullet$ 	& % РЗСК
	$\bullet$ 	& % К400 ВЧ
	$\bullet$ 	& % К400 ОПТИКА 
	$\bullet$ 	& % К400 КОЛЬЦО 
	Индикация предупреждения.	 
	\tabularnewline \hline
	
	203			& % Адрес 101
	2203		& % Адрес 104 (аппарат 1)
	3203		& % Адрес 104 (аппарат 2)
			 	& % Р400
	$\bullet$ 	& % РЗСК
	$\bullet$ 	& % К400 ВЧ
	$\bullet$ 	& % К400 ОПТИКА 
	$\bullet$ 	& % К400 КОЛЬЦО
	Индикация команд передатчика.	 
	\tabularnewline \hline
	
	204			& % Адрес 101
	2204		& % Адрес 104 (аппарат 1)
	3204		& % Адрес 104 (аппарат 2)
			 	& % Р400
	$\bullet$ 	& % РЗСК
	$\bullet$ 	& % К400 ВЧ
	$\bullet$ 	& % К400 ОПТИКА 
	$\bullet$ 	& % К400 КОЛЬЦО
	Индикация команд приемника.	
	\tabularnewline \hline
	
	301  302 ...  316	& % Адрес 101
	2301 2302 ... 2316	& % Адрес 104 (аппарат 1)
	3301 3302 ... 3316	& % Адрес 104 (аппарат 2)
	$\bullet$ 	& % Р400
	$\bullet$ 	& % РЗСК
	$\bullet$ 	& % К400 ВЧ
	$\bullet$ 	& % К400 ОПТИКА 
	$\bullet$ 	& % К400 КОЛЬЦО
	Индикация общей неисправности с кодом 0х0001, 0x0002 \ldots 0x8000.	 
	\tabularnewline \hline 
	 
	\lasthline
\end{tabularx}

Информация по неисправностям и предупрежденям приведена в Д-РО на соотвествующий аппарат. 






\end{document}

