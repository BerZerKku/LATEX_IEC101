\documentclass[russian,utf8,pointsection]{eskdtext}
\usepackage{cmap}			% Поиск по русским словам в конечном pdf документе
\usepackage{eskdchngsheet}
\usepackage[T2A]{fontenc}
\usepackage{pscyr}			% Подключение "красивых" шрифтов киррилицы
\usepackage{amstext}
\usepackage{amsmath}
\usepackage{listings}
\usepackage{color}
\usepackage{ifthen}

% работа с сылками
\usepackage{hyperref}
% \usepackage[usenames,dvipsnames,svgnames,table,rgb]{xcolor} 
\hypersetup{			
	unicode=true,           				% русские буквы в разделе PDF
	pdftitle={Аппаратура АВАНТ},			% Заголовок
	pdfauthor={Щеблыкин М.В.}, 				% Автор
	pdfsubject={Руководство по использованию протокола МЭК 870-5-101},	% Тема 
	colorlinks=true,      					% false: ссылки в рамках; true: цветные ссылки 
	linkcolor=black,        				% внутренние ссылки
	citecolor=blue,        					% на библиографию
	filecolor=blue,        					% на файлы
	urlcolor =blue          				% на URL 
}

% Дает доступ к командам:
% \MakeTextUppercase{} - сделать все символы заглавными
\usepackage{textcase} 

%Изменение отображения Содержания
%\makeatletter
%\renewcommand{\l@section}{\@dottedtocline{1}{0em}{1.25em}}
%\renewcommand{\l@subsection}{\@dottedtocline{2}{1.25em}{1.75em}}
%\renewcommand{\l@subsubsection}{\@dottedtocline{3}{2.75em}{2.6em}}
%\makeatother

% Работа с таблицами
% p{} - top align, m{} - middle align, b{} - bottom align
\usepackage{ltablex} 										% longtable с функциональностю tabularx
\usepackage{multirow} 										% Слияние строк в таблице
\renewcommand{\tabularxcolumn}[1]{>{\arraybackslash}m{#1}}	% выравнивание в ячейке таблицы по середине по вертикали
\newcolumntype{M}[1]{>{\centering \arraybackslash}m{#1}} 	% колонка с заданной шириной и выравниванием по центру
\newcolumntype{Z}{>{\centering \arraybackslash} X} 			% колонка с выравниванием по центру

% Заполнение граф Титульного листа и основной надписи
\ESKDcompany{ООО <<Прософт-Системы>>}
\ESKDclassCode{ОКП xx xxxx}
\ESKDtitle{Аппаратура <<АВАНТ>>}
\ESKDdocName{Руководство по использованию протокола МЭК 870-5-101}
\ESKDsignature{ПБКМ.424325.00x РЭ}
\ESKDgroup{\normalsize ООО <<Прософт-Системы>>}

% основная надпись
\ESKDauthor{\ESKDfontII Щеблыкин М.В.}
\ESKDchecker{\ESKDfontII Макаров Е.Г.}
%\ESKDnormContr{\ESKDfontII Бунина О.Ю.}
\ESKDapprovedBy{\ESKDfontII Чирков А.Г.}
\ESKDdate{2018/06/28}
\ESKDcolumnI{Аппаратура <<АВАНТ>> \\ \vspace{0.5cm} \ESKDfontIII{Руководство по использованию протокола МЭК870-5-101}}

% Добавлено отображение Города на Титульном листе
\renewcommand{\ESKDtheTitleFieldX}{Екатеринбург \\ \ESKDtheYear}

% Уменьшен размер шрифта для заголовков секций
\ESKDsectStyle{section}{\large \bfseries \MakeTextUppercase}

% Выравнивание по центру.
% Предназначено для выравнивания надписей в шапке таблицы
\newcommand{\calign}[1]{\centering #1 \arraybackslash} 

\newcommand{\adrY}{$\bullet$}
\newcommand{\adrN}{ }

\begin{document} 
	\maketitle
	\tableofcontents
	
	\newpage

	%%% ----------
\section{Общие сведения} \label{sec:overview}

К информационно-управляющей сети аппаратура <<АВАНТ>> может быть подключена с~помощью коммуникационного прота RS-485/RS-422 и~поддерживаемого протокола МЭК 870-5-101 (ГОСТ Р МЭК 870-5-101).  

Реализованный в аппаратуре <<АВАНТ>> протокол соотвествует стандартному протоколу МЭК 870-5-101.

В~сети МЭК 870-5-101 <<АВАНТ>> всегда выступает в~роли контролируемого пункта (КП) .

Реализованный протокол МЭК 870-5-101 обеспечивает считывание флагов приемопередатчика, список и~описание которых приведен в~приложении~\ref{tab:appMap101}.

Функции протокола реализованы в~блоке БСП (плата БСП-ПИ). Выводы подключения находятся на~клеммнике КВП.
	%%% ----------
\section{Технические характеристики} \label{sec:tth}

\begin{list}{--}{Технические характеристики:}
	\item Физический уровень: последовательный канал стандарта RS-485/RS-422. Цепи подключения линии гальванически развязаны с~остальными цепями и~корпусом аппарата (уровень изоляции 2500~эфф.).
	\item Максимальная длина линии <<пункт управления - <<АВАНТ>>>> определеятеся типом кабеля и~скоростью передачи. Рекомендуется использовать экранированную витую пару.
	\item Скорости обмена данными: 600, 1200, 2400, 4800, 9600, 19200~бит/с.
	\item Адреса устройства в~сети МЭК 870-5-101: от~1 до~247.
\end{list}

При передаче данных исползуется формат кадра FT1.2, определенный в ГОСТ~Р~МЭК~870-5-2. Допускается формат как с~фиксированной, так и с~переменной длиной блока.

Аппаратура <<АВАНТ>> поддерживает только небалансную передачу по каналу.

Для передачи данных используется только режим <<1>> (младший байт передается первым). 
 
В~случае обнаружения ошибки прием кадра прекращается. Для возобновления процесса приема кадров требуется, чтобы приемник зафиксировал, что входная линия перешла в~исходное состояние (уровень сигнала в~линии соответсвует уровню стопового бита) и~удерживается в~этом состоянии в~течение времени передачи не менее 33~бит.

\begin{list}{--}{Поддерживаются следующие стандартные идентификаторы типа (ASDU):}
	\item M\_SP\_NA\_1 1 Одноэлементная информация без~метки времени.
	\item M\_SP\_TB\_1 30 Одноэлементная информация с~меткой времени \\ СР56Время2а.
	\item M\_EI\_NA\_1 70 Окончение инициализации.
	\item C\_IC\_NA\_1 100 Команда опроса.
	\item C\_CS\_NA\_1 103 Команда синхронизации часов.
\end{list}

\begin{list}{--}{Фиксированные параметры пакета МЭК~870-5-101:} 
	\item Общий адрес ASDU : 1.
	\item Длина адреса станции (байт): 1.
	\item Длина общего адреса ASDU (байт): 1.
	\item Длина адреса объекта информации (байт): 2.
	\item Длина кода причины передачи (байт): 1.
\end{list}

Процедура опроса обеспечивается на~канальном уровне, который запрашивает пользовательские данные классов~1 и 2. 

Данные класса~2 можно считать командой опроса, при этом они будут переданы без метки времени.
	%%% ----------
\section{Установка параметров соединения} \label{sec:setup}

Параметры соединения должны быть настроены до~установки связи. 
 
Настройка параметров соединения производится с~пульта управления блока БСП в~меню <<Настройка/Интерфейс>>. Параметры соединения хранятся в~ПЗУ и не~требуют повторной настройки при~следующем включении питания.

\textbf{Изменение параметров соединения с~пульта управления при~установленном соединении приводит к~потере связи.}

В~таблице~\ref{tab:connection} приведены используемые параметры и их рекомендованные значения.

\begin{tabularx}{\linewidth}{| M{3cm} | X | M{4cm} | M{2.5cm} |}
	\caption{Параметры соединения} \label{tab:connection} \\
    
    \hline
    Название параметра			& 
    \calign{Описание параметра}	& 
    Возможные значения 			& 
    Значение \\ \hline 
    \endfirsthead
    
    \multicolumn{4}{r}{продолжение следует\ldots} \\  
    \endfoot 
	\endlastfoot
	
	\hline
	Название параметра 			&		
	\calign{Описание параметра}	&														
	Возможные значения			&
	Значение 					\\ \hline 
	\endhead
	
	Интерфейс связи				& 
	Выбор внешнего интерфейса связи.	& 
	USB, RS485 					& 
	RS485						\\ \hline
	
	Протокол					& 
	Текущий протокол связи.	 	& 
	Стандарт, MODBUS, МЭК-101	& 
	МЭК-101						\\ \hline
	
	Сетевой адрес (link)		& 
	Адрес аппарата в локальной сети.	& 
	от~1 до~247 				& 
	1							\\ \hline
	
	Биты данных					& 
	Число информационных бит в передаваемых и принимаемых байтах. & 
	8 							& 
	8~бит						\\ \hline
	
	Скорость передачи			& 
	Скорость передачи данных.	& 
	600, 1200, 2400, 4800, 9600, 19200 бит/с	& 
	19200						\\ \hline
	
	Четность					& 
	Схема контроля четности.	& 
	нет, чет, нечет			 	& 
	чет							\\ \hline
	
	Стоповые биты				& 
	Количество стоповых бит.	& 
	1 или 2						& 
	1~бит						\\ \hline
\end{tabularx}

Выбор используемого интерфейса RS-485/RS-422 осуществляется переключателями на~блоке КВП. Если аппарат является крайним в~цепи линии связи, то~рекомендуется подключить согласующий резистор номиналом 120~Ом (переключатель на блоке КВП). 


	
 	\ESKDappendix{Обязательное}{Карта памяти данных класса 2.} \label{app:map101_2}
 	%

\begin{tabularx}{\linewidth}{| M{2cm} | *{5}{p{0.3cm} |} X |}    
 	\caption{Карта памяти данных класса 2.}	\label{tab:appMap101Class2} 
	\tabularnewline
    
    \firsthline
    \multirow{2}{2cm}{\centering Адрес объекта}	& 
    \multicolumn{5}{c|}{Аппарат} 				&
    \centering \multirow{2}{*}{Описание} \tabularnewline \cline{2-6}
    											&
    \begin{sideways} Р400		\end{sideways}	&
    \begin{sideways} РЗСК ВЧ~	\end{sideways}	& 
    \begin{sideways} К400 ВЧ~	\end{sideways}	&
    \begin{sideways} ОПТИКА~ 	\end{sideways} 	&
    \begin{sideways} КОЛЬЦО~ 	\end{sideways} 	&
   	\tabularnewline \hline
    \endfirsthead

	\multicolumn{7}{l}{Продолжение таблицы~\ref{tab:appMap101Class2}}
	\tabularnewline \hline
	\multirow{2}{2cm}{\centering Адрес объекта}	& 
    \multicolumn{5}{c|}{Аппарат}				&
    \centering \multirow{2}{*}{Описание} \tabularnewline \cline{2-6}
    											&
    \begin{sideways} Р400		\end{sideways}	&
    \begin{sideways} РЗСК ВЧ~	\end{sideways}	& 
    \begin{sideways} К400 ВЧ~	\end{sideways}	&
    \begin{sideways} ОПТИКА~ 	\end{sideways} 	&
    \begin{sideways} КОЛЬЦО~ 	\end{sideways} 	&
   	\tabularnewline \hline
  	\endhead

    \multicolumn{7}{r}{продолжение следует\ldots}
	\endfoot
	\endlastfoot

	201		& \adrY	& \adrY	& \adrY	& \adrY	& \adrY	& Индикация неисправности.						\tabularnewline \hline
	202		& \adrY	& \adrY	& \adrY	& \adrY	& \adrY	& Индикация предупреждения.						\tabularnewline \hline
	203		& 		& \adrY	& \adrY	& \adrY	& \adrY	& Индикация команд передатчика.					\tabularnewline \hline
	204		& 		& \adrY	& \adrY	& \adrY	& \adrY	& Индикация команд приемника.					\tabularnewline \hline	
	---		&		&		& 		& 		& 		& --- 											\tabularnewline \hline
	301		& \adrY	& \adrY	& \adrY	& \adrY	& \adrY	& Индикация общей неисправности 0x0001.			\tabularnewline \hline 
	302		& \adrY	& \adrY	& \adrY	& \adrY	& \adrY	& Индикация общей неисправности 0x0002.			\tabularnewline \hline
	303		& \adrY	& \adrY	& \adrY	& \adrY	& \adrY	& Индикация общей неисправности 0x0004.			\tabularnewline \hline
	304		& \adrY	& \adrY	& \adrY	& \adrY	& \adrY	& Индикация общей неисправности 0x0008.			\tabularnewline \hline
	305		& \adrY	& \adrY	& \adrY	& \adrY	& \adrY	& Индикация общей неисправности 0x0010.			\tabularnewline \hline
	306		& \adrY	& \adrY	& \adrY	& \adrY	& \adrY	& Индикация общей неисправности 0x0020.			\tabularnewline \hline
	307		& \adrY	& \adrY	& \adrY	& \adrY	& \adrY	& Индикация общей неисправности 0x0040.			\tabularnewline \hline
	308		& \adrY	& \adrY	& \adrY	& \adrY	& \adrY	& Индикация общей неисправности 0x0080.			\tabularnewline \hline
	309		& \adrY	& \adrY	& \adrY	& \adrY	& \adrY	& Индикация общей неисправности 0x0100.			\tabularnewline \hline
	310		& \adrY	& \adrY	& \adrY	& \adrY	& \adrY	& Индикация общей неисправности 0x0200.			\tabularnewline \hline
	311		& \adrY	& \adrY	& \adrY	& \adrY	& \adrY	& Индикация общей неисправности 0x0400.			\tabularnewline \hline
	312		& \adrY	& \adrY	& \adrY	& \adrY	& \adrY	& Индикация общей неисправности 0x0800.			\tabularnewline \hline
	313		& \adrY	& \adrY	& \adrY	& \adrY	& \adrY	& Индикация общей неисправности 0x1000.			\tabularnewline \hline
	314		& \adrY	& \adrY	& \adrY	& \adrY	& \adrY	& Индикация общей неисправности 0x2000.			\tabularnewline \hline
	315		& \adrY	& \adrY	& \adrY	& \adrY	& \adrY	& Индикация общей неисправности 0x4000.			\tabularnewline \hline
	316		& \adrY	& \adrY	& \adrY	& \adrY	& \adrY	& Индикация общей неисправности 0x8000.			\tabularnewline \hline	
	317		& \adrY	& \adrY	& \adrY	& \adrY	& \adrY	& Индикация общего предупреждения 0x0001.		\tabularnewline \hline 
	318		& \adrY	& \adrY	& \adrY	& \adrY	& \adrY	& Индикация общего предупреждения 0x0002.		\tabularnewline \hline 
 	319		& \adrY	& \adrY	& \adrY	& \adrY	& \adrY	& Индикация общего предупреждения 0x0004.		\tabularnewline \hline 
 	320		& \adrY	& \adrY	& \adrY	& \adrY	& \adrY	& Индикация общего предупреждения 0x0008.		\tabularnewline \hline 
 	321		& \adrY	& \adrY	& \adrY	& \adrY	& \adrY	& Индикация общего предупреждения 0x0010.		\tabularnewline \hline 
 	322		& \adrY	& \adrY	& \adrY	& \adrY	& \adrY	& Индикация общего предупреждения 0x0020.		\tabularnewline \hline
 	323		& \adrY	& \adrY	& \adrY	& \adrY	& \adrY	& Индикация общего предупреждения 0x0040.		\tabularnewline \hline
 	324		& \adrY	& \adrY	& \adrY	& \adrY	& \adrY	& Индикация общего предупреждения 0x0080.		\tabularnewline \hline
 	325		& \adrY	& \adrY	& \adrY	& \adrY	& \adrY	& Индикация общего предупреждения 0x0100.		\tabularnewline \hline
 	326		& \adrY	& \adrY	& \adrY	& \adrY	& \adrY	& Индикация общего предупреждения 0x0200.		\tabularnewline \hline
 	327		& \adrY	& \adrY	& \adrY	& \adrY	& \adrY	& Индикация общего предупреждения 0x0400.		\tabularnewline \hline
 	328		& \adrY	& \adrY	& \adrY	& \adrY	& \adrY	& Индикация общего предупреждения 0x0800.		\tabularnewline \hline
 	329		& \adrY	& \adrY	& \adrY	& \adrY	& \adrY	& Индикация общего предупреждения 0x1000.		\tabularnewline \hline
 	330		& \adrY	& \adrY	& \adrY	& \adrY	& \adrY	& Индикация общего предупреждения 0x2000.		\tabularnewline \hline
 	331		& \adrY	& \adrY	& \adrY	& \adrY	& \adrY	& Индикация общего предупреждения 0x4000.		\tabularnewline \hline
	332		& \adrY	& \adrY	& \adrY	& \adrY	& \adrY	& Индикация общего предупреждения 0x8000.		\tabularnewline \hline 
	---	 	&		&		& 		& 		& 		& --- 											\tabularnewline \hline 
	400		& 		& \adrY	& \adrY	& \adrY & \adrY	& Передатчик подключен.							\tabularnewline \hline
	401		& 		& \adrY	& \adrY	& \adrY	& \adrY	& Индикация неисправности передатчика 0x0001.	\tabularnewline \hline 
	402		& 		& \adrY	& \adrY	& \adrY	& \adrY	& Индикация неисправности передатчика 0x0002.	\tabularnewline \hline
	403		& 		& \adrY	& \adrY	& \adrY	& \adrY	& Индикация неисправности передатчика 0x0004.	\tabularnewline \hline
	404		& 		& \adrY	& \adrY	& \adrY	& \adrY	& Индикация неисправности передатчика 0x0008.	\tabularnewline \hline
	405		& 		& \adrY	& \adrY	& \adrY	& \adrY	& Индикация неисправности передатчика 0x0010.	\tabularnewline \hline
	406		& 		& \adrY	& \adrY	& \adrY	& \adrY	& Индикация неисправности передатчика 0x0020.	\tabularnewline \hline
	407		& 		& \adrY	& \adrY	& \adrY	& \adrY	& Индикация неисправности передатчика 0x0040.	\tabularnewline \hline
	408		& 		& \adrY	& \adrY	& \adrY	& \adrY	& Индикация неисправности передатчика 0x0080.	\tabularnewline \hline
	409		& 		& \adrY	& \adrY	& \adrY	& \adrY	& Индикация неисправности передатчика 0x0100.	\tabularnewline \hline
	410		& 		& \adrY	& \adrY	& \adrY	& \adrY	& Индикация неисправности передатчика 0x0200.	\tabularnewline \hline
	411		& 		& \adrY	& \adrY	& \adrY	& \adrY	& Индикация неисправности передатчика 0x0400.	\tabularnewline \hline
	412		& 		& \adrY	& \adrY	& \adrY	& \adrY	& Индикация неисправности передатчика 0x0800.	\tabularnewline \hline
	413		& 		& \adrY	& \adrY	& \adrY	& \adrY	& Индикация неисправности передатчика 0x1000.	\tabularnewline \hline
	414		& 		& \adrY	& \adrY	& \adrY	& \adrY	& Индикация неисправности передатчика 0x2000.	\tabularnewline \hline
	415		& 		& \adrY	& \adrY	& \adrY	& \adrY	& Индикация неисправности передатчика 0x4000.	\tabularnewline \hline
	416		& 		& \adrY	& \adrY	& \adrY	& \adrY	& Индикация неисправности передатчика 0x8000.	\tabularnewline \hline
	417		& 		& \adrY	& \adrY	& \adrY	& \adrY	& Индикация предупреждения передатчика 0x0001.	\tabularnewline \hline 
	418		& 		& \adrY	& \adrY	& \adrY	& \adrY	& Индикация предупреждения передатчика 0x0002.	\tabularnewline \hline 
	419		& 		& \adrY	& \adrY	& \adrY	& \adrY	& Индикация предупреждения передатчика 0x0004.	\tabularnewline \hline
	420		& 		& \adrY	& \adrY	& \adrY	& \adrY	& Индикация предупреждения передатчика 0x0008.	\tabularnewline \hline
	421		& 		& \adrY	& \adrY	& \adrY	& \adrY	& Индикация предупреждения передатчика 0x0010.	\tabularnewline \hline
	422		& 		& \adrY	& \adrY	& \adrY	& \adrY	& Индикация предупреждения передатчика 0x0020.	\tabularnewline \hline
	423		& 		& \adrY	& \adrY	& \adrY	& \adrY	& Индикация предупреждения передатчика 0x0040.	\tabularnewline \hline
	424		& 		& \adrY	& \adrY	& \adrY	& \adrY	& Индикация предупреждения передатчика 0x0080.	\tabularnewline \hline
	425		& 		& \adrY	& \adrY	& \adrY	& \adrY	& Индикация предупреждения передатчика 0x0100.	\tabularnewline \hline
	426		& 		& \adrY	& \adrY	& \adrY	& \adrY	& Индикация предупреждения передатчика 0x0200.	\tabularnewline \hline
	427		& 		& \adrY	& \adrY	& \adrY	& \adrY	& Индикация предупреждения передатчика 0x0400.	\tabularnewline \hline
	428		& 		& \adrY	& \adrY	& \adrY	& \adrY	& Индикация предупреждения передатчика 0x0800.	\tabularnewline \hline
	429		& 		& \adrY	& \adrY	& \adrY	& \adrY	& Индикация предупреждения передатчика 0x1000.	\tabularnewline \hline
	430		& 		& \adrY	& \adrY	& \adrY	& \adrY	& Индикация предупреждения передатчика 0x2000.	\tabularnewline \hline
	431		& 		& \adrY	& \adrY	& \adrY	& \adrY	& Индикация предупреждения передатчика 0x4000.	\tabularnewline \hline
	432		& 		& \adrY	& \adrY	& \adrY	& \adrY	& Индикация предупреждения передатчика 0x8000.	\tabularnewline \hline 
	--- 	&		&		& 		& 		& 		& --- 											\tabularnewline \hline 	
	450		& 	 	& \adrY	& \adrY	& \adrY	& \adrY	& Индикация команды передатчика 1. 				\tabularnewline \hline 
	451		& 		& \adrY	& \adrY	& \adrY	& \adrY	& Индикация команды передатчика 2. 				\tabularnewline \hline
	452		& 		& \adrY	& \adrY	& \adrY	& \adrY	& Индикация команды передатчика 3. 				\tabularnewline \hline
	453		&		& \adrY	& \adrY	& \adrY	& \adrY	& Индикация команды передатчика 4. 				\tabularnewline \hline
	454		& 		& 		& \adrY	& \adrY	& \adrY	& Индикация команды передатчика 5. 				\tabularnewline \hline
	455		& 		& 		& \adrY	& \adrY	& \adrY	& Индикация команды передатчика 6. 				\tabularnewline \hline
	456		& 		& 		& \adrY	& \adrY	& \adrY	& Индикация команды передатчика 7. 				\tabularnewline \hline
	457		& 		& 		& \adrY	& \adrY	& \adrY	& Индикация команды передатчика 8. 				\tabularnewline \hline
	458		& 		& 		& \adrY	& \adrY	& \adrY	& Индикация команды передатчика 9. 				\tabularnewline \hline
	459		& 		& 		& \adrY	& \adrY	& \adrY	& Индикация команды передатчика 10. 			\tabularnewline \hline
	460		& 		& 		& \adrY	& \adrY	& \adrY	& Индикация команды передатчика 11. 			\tabularnewline \hline
	461		& 		& 		& \adrY	& \adrY	& \adrY	& Индикация команды передатчика 12. 			\tabularnewline \hline
	462		& 		& 		& \adrY	& \adrY	& \adrY	& Индикация команды передатчика 13. 			\tabularnewline \hline
	463		& 		& 		& \adrY	& \adrY	& \adrY	& Индикация команды передатчика 14. 			\tabularnewline \hline
	464		& 		& 		& \adrY	& \adrY	& \adrY	& Индикация команды передатчика 15. 			\tabularnewline \hline
	465		& 		& 		& \adrY	& \adrY	& \adrY	& Индикация команды передатчика 16. 			\tabularnewline \hline
	466		& 		& 		& \adrY	& \adrY	& \adrY	& Индикация команды передатчика 17. 			\tabularnewline \hline
	467		& 		& 		& \adrY	& \adrY	& \adrY	& Индикация команды передатчика 18. 			\tabularnewline \hline
	468		& 		& 		& \adrY	& \adrY	& \adrY	& Индикация команды передатчика 19. 			\tabularnewline \hline
	469		& 		& 		& \adrY	& \adrY	& \adrY	& Индикация команды передатчика 20. 			\tabularnewline \hline
	470		& 		& 		& \adrY	& \adrY	& \adrY	& Индикация команды передатчика 21. 			\tabularnewline \hline
	471		& 		& 		& \adrY	& \adrY	& \adrY	& Индикация команды передатчика 22. 			\tabularnewline \hline
	472		& 		& 		& \adrY	& \adrY	& \adrY	& Индикация команды передатчика 23. 			\tabularnewline \hline
	473		& 		& 		& \adrY	& \adrY	& \adrY	& Индикация команды передатчика 24. 			\tabularnewline \hline
	474		& 		& 		& \adrY	& \adrY	& \adrY	& Индикация команды передатчика 25. 			\tabularnewline \hline
	475		& 		& 		& \adrY	& \adrY	& \adrY	& Индикация команды передатчика 26. 			\tabularnewline \hline
	476		& 		& 		& \adrY	& \adrY	& \adrY	& Индикация команды передатчика 27. 			\tabularnewline \hline
	477		& 		& 		& \adrY	& \adrY	& \adrY	& Индикация команды передатчика 28. 			\tabularnewline \hline
	478		& 		& 		& \adrY	& \adrY	& \adrY	& Индикация команды передатчика 29. 			\tabularnewline \hline
	479		& 		& 		& \adrY	& \adrY	& \adrY	& Индикация команды передатчика 30. 			\tabularnewline \hline
	480		& 		& 		& \adrY	& \adrY	& \adrY	& Индикация команды передатчика 31. 			\tabularnewline \hline
	481		&		&		& \adrY	& \adrY	& \adrY	& Индикация команды передатчика 32. 			\tabularnewline \hline
	--- 	&		&		& 		& 		& 		& --- 											\tabularnewline \hline 
	485		& 		& \adrY	& \adrY	& \adrY	& \adrY	& Индикация режима передатчика <<Выведен>>. 	\tabularnewline \hline
	486		& 		& \adrY	& \adrY	& \adrY	& \adrY	& Индикация режима передатчика <<Введен>>. 		\tabularnewline \hline
	487		& 		& \adrY	& \adrY	& \adrY	& \adrY	& Индикация режима передатчика <<Тест>>. 		\tabularnewline \hline
	--- 	&		&		& 		& 		& 		& --- 											\tabularnewline \hline 
	500		& 		& \adrY	& \adrY	& \adrY & \adrY	& Приемник подключен.							\tabularnewline \hline	
	501		& 		& \adrY	& \adrY	& \adrY	& \adrY	& Индикация неисправности приемника 0x0001.		\tabularnewline \hline 
	502		& 		& \adrY	& \adrY	& \adrY	& \adrY	& Индикация неисправности приемника 0x0002.		\tabularnewline \hline
	503		& 		& \adrY	& \adrY	& \adrY	& \adrY	& Индикация неисправности приемника 0x0004.		\tabularnewline \hline
	504		& 		& \adrY	& \adrY	& \adrY	& \adrY	& Индикация неисправности приемника 0x0008.		\tabularnewline \hline
	505		& 		& \adrY	& \adrY	& \adrY	& \adrY	& Индикация неисправности приемника 0x0010.		\tabularnewline \hline
	506		& 		& \adrY	& \adrY	& \adrY	& \adrY	& Индикация неисправности приемника 0x0020.		\tabularnewline \hline
	507		& 		& \adrY	& \adrY	& \adrY	& \adrY	& Индикация неисправности приемника 0x0040.		\tabularnewline \hline
	508		& 		& \adrY	& \adrY	& \adrY	& \adrY	& Индикация неисправности приемника 0x0080.		\tabularnewline \hline
	509		& 		& \adrY	& \adrY	& \adrY	& \adrY	& Индикация неисправности приемника 0x0100.		\tabularnewline \hline
	510		& 		& \adrY	& \adrY	& \adrY	& \adrY	& Индикация неисправности приемника 0x0200.		\tabularnewline \hline
	511		& 		& \adrY	& \adrY	& \adrY	& \adrY	& Индикация неисправности приемника 0x0400.		\tabularnewline \hline
	512		& 		& \adrY	& \adrY	& \adrY	& \adrY	& Индикация неисправности приемника 0x0800.		\tabularnewline \hline
	513		& 		& \adrY	& \adrY	& \adrY	& \adrY	& Индикация неисправности приемника 0x1000.		\tabularnewline \hline
	514		& 		& \adrY	& \adrY	& \adrY	& \adrY	& Индикация неисправности приемника 0x2000.		\tabularnewline \hline
	515		& 		& \adrY	& \adrY	& \adrY	& \adrY	& Индикация неисправности приемника 0x4000.		\tabularnewline \hline
	516		& 		& \adrY	& \adrY	& \adrY	& \adrY	& Индикация неисправности приемника 0x8000.		\tabularnewline \hline
	517		& 		& \adrY	& \adrY	& \adrY	& \adrY	& Индикация предупреждения приемника 0x0001.	\tabularnewline \hline 
	518		& 		& \adrY	& \adrY	& \adrY	& \adrY	& Индикация предупреждения приемника 0x0002.	\tabularnewline \hline 
	519		& 		& \adrY	& \adrY	& \adrY	& \adrY	& Индикация предупреждения приемника 0x0004.	\tabularnewline \hline
	520		& 		& \adrY	& \adrY	& \adrY	& \adrY	& Индикация предупреждения приемника 0x0008.	\tabularnewline \hline
	521		& 		& \adrY	& \adrY	& \adrY	& \adrY	& Индикация предупреждения приемника 0x0010.	\tabularnewline \hline
	522		& 		& \adrY	& \adrY	& \adrY	& \adrY	& Индикация предупреждения приемника 0x0020.	\tabularnewline \hline
	523		& 		& \adrY	& \adrY	& \adrY	& \adrY	& Индикация предупреждения приемника 0x0040.	\tabularnewline \hline
	524		& 		& \adrY	& \adrY	& \adrY	& \adrY	& Индикация предупреждения приемника 0x0080.	\tabularnewline \hline
	525		& 		& \adrY	& \adrY	& \adrY	& \adrY	& Индикация предупреждения приемника 0x0100.	\tabularnewline \hline
	526		& 		& \adrY	& \adrY	& \adrY	& \adrY	& Индикация предупреждения приемника 0x0200.	\tabularnewline \hline
	527		& 		& \adrY	& \adrY	& \adrY	& \adrY	& Индикация предупреждения приемника 0x0400.	\tabularnewline \hline
	528		& 		& \adrY	& \adrY	& \adrY	& \adrY	& Индикация предупреждения приемника 0x0800.	\tabularnewline \hline
	529		& 		& \adrY	& \adrY	& \adrY	& \adrY	& Индикация предупреждения приемника 0x1000.	\tabularnewline \hline
	530		& 		& \adrY	& \adrY	& \adrY	& \adrY	& Индикация предупреждения приемника 0x2000.	\tabularnewline \hline
	531		& 		& \adrY	& \adrY	& \adrY	& \adrY	& Индикация предупреждения приемника 0x4000.	\tabularnewline \hline
	532		& 		& \adrY	& \adrY	& \adrY	& \adrY	& Индикация предупреждения приемника 0x8000.	\tabularnewline \hline 
	--- 	&		&		& 		& 		& 		& --- 											\tabularnewline \hline 	
	550		& 	 	& \adrY	& \adrY	& \adrY	& \adrY	& Индикация команды приемника 1. 				\tabularnewline \hline 
	551		& 		& \adrY	& \adrY	& \adrY	& \adrY	& Индикация команды приемника 2. 				\tabularnewline \hline
	552		& 		& \adrY	& \adrY	& \adrY	& \adrY	& Индикация команды приемника 3. 				\tabularnewline \hline
	553		&		& \adrY	& \adrY	& \adrY	& \adrY	& Индикация команды приемника 4. 				\tabularnewline \hline
	554		& 		& 		& \adrY	& \adrY	& \adrY	& Индикация команды приемника 5. 				\tabularnewline \hline
	555		& 		& 		& \adrY	& \adrY	& \adrY	& Индикация команды приемника 6. 				\tabularnewline \hline
	556		& 		& 		& \adrY	& \adrY	& \adrY	& Индикация команды приемника 7. 				\tabularnewline \hline
	557		& 		& 		& \adrY	& \adrY	& \adrY	& Индикация команды приемника 8. 				\tabularnewline \hline
	558		& 		& 		& \adrY	& \adrY	& \adrY	& Индикация команды приемника 9. 				\tabularnewline \hline
	559		& 		& 		& \adrY	& \adrY	& \adrY	& Индикация команды приемника 10. 				\tabularnewline \hline
	560		& 		& 		& \adrY	& \adrY	& \adrY	& Индикация команды приемника 11. 				\tabularnewline \hline
	561		& 		& 		& \adrY	& \adrY	& \adrY	& Индикация команды приемника 12. 				\tabularnewline \hline
	562		& 		& 		& \adrY	& \adrY	& \adrY	& Индикация команды приемника 13. 				\tabularnewline \hline
	563		& 		& 		& \adrY	& \adrY	& \adrY	& Индикация команды приемника 14. 				\tabularnewline \hline
	564		& 		& 		& \adrY	& \adrY	& \adrY	& Индикация команды приемника 15. 				\tabularnewline \hline
	565		& 		& 		& \adrY	& \adrY	& \adrY	& Индикация команды приемника 16. 				\tabularnewline \hline
	566		& 		& 		& \adrY	& \adrY	& \adrY	& Индикация команды приемника 17. 				\tabularnewline \hline
	567		& 		& 		& \adrY	& \adrY	& \adrY	& Индикация команды приемника 18. 				\tabularnewline \hline
	568		& 		& 		& \adrY	& \adrY	& \adrY	& Индикация команды приемника 19. 				\tabularnewline \hline
	569		& 		& 		& \adrY	& \adrY	& \adrY	& Индикация команды приемника 20. 				\tabularnewline \hline
	570		& 		& 		& \adrY	& \adrY	& \adrY	& Индикация команды приемника 21. 				\tabularnewline \hline
	571		& 		& 		& \adrY	& \adrY	& \adrY	& Индикация команды приемника 22. 				\tabularnewline \hline
	572		& 		& 		& \adrY	& \adrY	& \adrY	& Индикация команды приемника 23. 				\tabularnewline \hline
	573		& 		& 		& \adrY	& \adrY	& \adrY	& Индикация команды приемника 24. 				\tabularnewline \hline
	574		& 		& 		& \adrY	& \adrY	& \adrY	& Индикация команды приемника 25. 				\tabularnewline \hline
	575		& 		& 		& \adrY	& \adrY	& \adrY	& Индикация команды приемника 26. 				\tabularnewline \hline
	576		& 		& 		& \adrY	& \adrY	& \adrY	& Индикация команды приемника 27. 				\tabularnewline \hline
	577		& 		& 		& \adrY	& \adrY	& \adrY	& Индикация команды приемника 28. 				\tabularnewline \hline
	578		& 		& 		& \adrY	& \adrY	& \adrY	& Индикация команды приемника 29. 				\tabularnewline \hline
	579		& 		& 		& \adrY	& \adrY	& \adrY	& Индикация команды приемника 30. 				\tabularnewline \hline
	580		& 		& 		& \adrY	& \adrY	& \adrY	& Индикация команды приемника 31. 				\tabularnewline \hline
	581		&		&		& \adrY	& \adrY	& \adrY	& Индикация команды приемника 32. 				\tabularnewline \hline
	--- 	&		&		& 		& 		& 		& --- 											\tabularnewline \hline  
	585		& 		& \adrY	& \adrY	& \adrY	& \adrY	& Индикация режима приемника <<Выведен>>.		\tabularnewline \hline
	587		& 		& \adrY	& \adrY	& \adrY	& \adrY	& Индикация режима приемника <<Готов>>. 		\tabularnewline \hline
	586		& 		& \adrY	& \adrY	& \adrY	& \adrY	& Индикация режима приемника <<Введен>>.		\tabularnewline \hline
	587		& 		& \adrY	& \adrY	& \adrY	& \adrY	& Индикация режима приемника <<Тест>>. 			\tabularnewline \hline	
	--- 	&		&		& 		& 		& 		& --- 											\tabularnewline \hline
	600		& \adrY	& \adrY	& 		& 		& 		& Защита подключена.							\tabularnewline \hline
	601		& \adrY	& \adrY	& 		& 		& 		& Индикация неисправности защиты 0x0001.		\tabularnewline \hline 
	602		& \adrY	& \adrY	& 		& 		& 		& Индикация неисправности защиты 0x0002.		\tabularnewline \hline
	603		& \adrY	& \adrY	& 		& 		& 		& Индикация неисправности защиты 0x0004.		\tabularnewline \hline
	604		& \adrY	& \adrY	& 		& 		& 		& Индикация неисправности защиты 0x0008.		\tabularnewline \hline
	605		& \adrY	& \adrY	& 		& 		& 		& Индикация неисправности защиты 0x0010.		\tabularnewline \hline
	606		& \adrY	& \adrY	& 		& 		& 		& Индикация неисправности защиты 0x0020.		\tabularnewline \hline
	607		& \adrY	& \adrY	& 		& 		& 		& Индикация неисправности защиты 0x0040.		\tabularnewline \hline
	608		& \adrY	& \adrY	& 		& 		& 		& Индикация неисправности защиты 0x0080.		\tabularnewline \hline
	609		& \adrY	& \adrY	& 		& 		& 		& Индикация неисправности защиты 0x0100.		\tabularnewline \hline
	610		& \adrY	& \adrY	& 		& 		& 		& Индикация неисправности защиты 0x0200.		\tabularnewline \hline
	611		& \adrY	& \adrY	& 		& 		& 		& Индикация неисправности защиты 0x0400.		\tabularnewline \hline
	612		& \adrY	& \adrY	& 		& 		& 		& Индикация неисправности защиты 0x0800.		\tabularnewline \hline
	613		& \adrY	& \adrY	& 		& 		& 		& Индикация неисправности защиты 0x1000.		\tabularnewline \hline
	614		& \adrY	& \adrY	& 		& 		& 		& Индикация неисправности защиты 0x2000.		\tabularnewline \hline
	615		& \adrY	& \adrY	& 		& 		& 		& Индикация неисправности защиты 0x4000.		\tabularnewline \hline
	616		& \adrY	& \adrY	& 		& 		& 		& Индикация неисправности защиты 0x8000.		\tabularnewline \hline
	617		& \adrY	& \adrY	& 		& 		& 		& Индикация предупреждения защиты 0x0001.		\tabularnewline \hline 
	618		& \adrY	& \adrY	& 		& 		& 		& Индикация предупреждения защиты 0x0002.		\tabularnewline \hline 
	619		& \adrY	& \adrY	& 		& 		& 		& Индикация предупреждения защиты 0x0004.		\tabularnewline \hline
	620		& \adrY	& \adrY	& 		& 		& 		& Индикация предупреждения защиты 0x0008.		\tabularnewline \hline
	621		& \adrY	& \adrY	& 		& 		& 		& Индикация предупреждения защиты 0x0010.		\tabularnewline \hline
	622		& \adrY	& \adrY	& 		& 		& 		& Индикация предупреждения защиты 0x0020.		\tabularnewline \hline
	623		& \adrY	& \adrY	& 		& 		& 		& Индикация предупреждения защиты 0x0040.		\tabularnewline \hline
	624		& \adrY	& \adrY	& 		& 		& 		& Индикация предупреждения защиты 0x0080.		\tabularnewline \hline
	625		& \adrY	& \adrY	& 		& 		& 		& Индикация предупреждения защиты 0x0100.		\tabularnewline \hline
	626		& \adrY	& \adrY	& 		& 		& 		& Индикация предупреждения защиты 0x0200.		\tabularnewline \hline
	627		& \adrY	& \adrY	& 		& 		& 		& Индикация предупреждения защиты 0x0400.		\tabularnewline \hline
	628		& \adrY	& \adrY	& 		& 		& 		& Индикация предупреждения защиты 0x0800.		\tabularnewline \hline	
	629		& \adrY	& \adrY	& 		& 		& 		& Индикация предупреждения защиты 0x1000.		\tabularnewline \hline
	630		& \adrY	& \adrY	& 		& 		& 		& Индикация предупреждения защиты 0x2000.		\tabularnewline \hline
	631		& \adrY	& \adrY	& 		& 		& 		& Индикация предупреждения защиты 0x4000.		\tabularnewline \hline
	632		& \adrY	& \adrY	& 		& 		& 		& Индикация предупреждения защиты 0x8000.		\tabularnewline \hline
	633		& \adrY	& \adrY	& 		& 		& 		& Пуск.											\tabularnewline \hline
	634		& \adrY	& \adrY	& 		& 		& 		& Останов.										\tabularnewline \hline
	635		& \adrY	& \adrY	& 		& 		& 		& Манипуляция.									\tabularnewline \hline
	636		& \adrY	& \adrY	& 		& 		& 		& Передача.										\tabularnewline \hline
	637		& \adrY	& \adrY	& 		& 		& 		& Прием.										\tabularnewline

	\lasthline
\end{tabularx}




 
 	
 	\ESKDappendix{Обязательное}{Карта памяти данных класса 1.} \label{app:map101_1}
 	%

\begin{tabularx}{\linewidth}{| M{2cm} | *{5}{p{0.3cm} |} X |}    
 	\caption{Карта памяти данных класса 1.}	\label{tab:appMap101Class1} 
	\tabularnewline
    
    \firsthline
    \multirow{2}{2cm}{\centering Адрес объекта}	& 
    \multicolumn{5}{c|}{Аппарат} 				&
    \centering \multirow{2}{*}{Описание} \tabularnewline \cline{2-6}
    											&
    \begin{sideways} Р400		\end{sideways}	&
    \begin{sideways} РЗСК ВЧ~	\end{sideways}	& 
    \begin{sideways} К400 ВЧ~	\end{sideways}	&
    \begin{sideways} ОПТИКА~ 	\end{sideways} 	&
    \begin{sideways} КОЛЬЦО~ 	\end{sideways} 	&
   	\tabularnewline \hline
    \endfirsthead

	\multicolumn{7}{l}{Продолжение таблицы~\ref{tab:appMap101Class1}}
	\tabularnewline \hline
	\multirow{2}{2cm}{\centering Адрес объекта}	& 
    \multicolumn{5}{c|}{Аппарат}				&
    \centering \multirow{2}{*}{Описание} \tabularnewline \cline{2-6} 
    											&
    \begin{sideways} Р400		\end{sideways}	&
    \begin{sideways} РЗСК ВЧ~	\end{sideways}	& 
    \begin{sideways} К400 ВЧ~	\end{sideways}	&
    \begin{sideways} ОПТИКА~ 	\end{sideways} 	&
    \begin{sideways} КОЛЬЦО~ 	\end{sideways} 	& 
   	\tabularnewline \hline
  	\endhead

    \multicolumn{7}{r}{продолжение следует\ldots}
	\endfoot
	\endlastfoot
	
	\multicolumn{7}{|c|}{Журнал событий.} 							\tabularnewline \hline
	700		& \adrY	& \adrY	& \adrY	& \adrY	& \adrY	& Событие №1.	\tabularnewline \hline
	701		& \adrY	& \adrY	& \adrY	& \adrY	& \adrY	& Событие №2.	\tabularnewline \hline
	702		& \adrY	& \adrY	& \adrY	& \adrY	& \adrY	& Событие №3.	\tabularnewline \hline
	703		& \adrY	& \adrY	& \adrY	& \adrY	& \adrY	& Событие №4.	\tabularnewline \hline
	704		& \adrY	& \adrY	& \adrY	& \adrY	& \adrY	& Событие №5.	\tabularnewline \hline
	705		& \adrY	& \adrY	& \adrY	& \adrY	& \adrY	& Событие №6.	\tabularnewline \hline
	706		& \adrY	& \adrY	& \adrY	& \adrY	& \adrY	& Событие №7.	\tabularnewline \hline
	707		& \adrY	& \adrY	& \adrY	& \adrY	& \adrY	& Событие №8.	\tabularnewline \hline
	708		& \adrY	& \adrY	& \adrY	& \adrY	& \adrY	& Событие №9.	\tabularnewline \hline
	709		& \adrY	& \adrY	& \adrY	& \adrY	& \adrY	& Событие №10.	\tabularnewline \hline
	710		& \adrY	& \adrY	& \adrY	& \adrY	& \adrY	& Событие №11.	\tabularnewline \hline
	711		& \adrY	& \adrY	& \adrY	& \adrY	& \adrY	& Событие №12.	\tabularnewline \hline
	712		& \adrY	& \adrY	& \adrY	& \adrY	& \adrY	& Событие №13.	\tabularnewline \hline
	713		& \adrY	& \adrY	& \adrY	& \adrY	& \adrY	& Событие №14.	\tabularnewline \hline
	714		& \adrY	& \adrY	& \adrY	& \adrY	& \adrY	& Событие №15.	\tabularnewline \hline
	715		& \adrY	& \adrY	& \adrY	& \adrY	& \adrY	& Событие №16.	\tabularnewline \hline
	716		& \adrY	& \adrY	& \adrY	& \adrY	& \adrY	& Событие №17.	\tabularnewline \hline
	717		& \adrY	& \adrY	& \adrY	& \adrY	& \adrY	& Событие №18.	\tabularnewline \hline
	718		& \adrY	& \adrY	& \adrY	& \adrY	& \adrY	& Событие №19.	\tabularnewline \hline
	719		& \adrY	& \adrY	& \adrY	& \adrY	& \adrY	& Событие №20.	\tabularnewline \hline
	720		& \adrY	& \adrY	& \adrY	& \adrY	& \adrY	& Событие №21.	\tabularnewline \hline
	721		& \adrY	& \adrY	& \adrY	& \adrY	& \adrY	& Событие №22.	\tabularnewline \hline
	722		& \adrY	& \adrY	& \adrY	& \adrY	& \adrY	& Событие №23.	\tabularnewline \hline
	723		& \adrY	& \adrY	& \adrY	& \adrY	& \adrY	& Событие №24.	\tabularnewline \hline
	724		& \adrY	& \adrY	& \adrY	& \adrY	& \adrY	& Событие №25.	\tabularnewline \hline
	725		& \adrY	& \adrY	& \adrY	& \adrY	& \adrY	& Событие №26.	\tabularnewline \hline
	726		& \adrY	& \adrY	& \adrY	& \adrY	& \adrY	& Событие №27.	\tabularnewline \hline
	727		& \adrY	& \adrY	& \adrY	& \adrY	& \adrY	& Событие №28.	\tabularnewline \hline
	728		& \adrY	& \adrY	& \adrY	& \adrY	& \adrY	& Событие №29.	\tabularnewline \hline
	729		& \adrY	& \adrY	& \adrY	& \adrY	& \adrY	& Событие №30.	\tabularnewline \hline
	730		& \adrY	& \adrY	& \adrY	& \adrY	& \adrY	& Событие №31.	\tabularnewline \hline
	731		& \adrY	& \adrY	& \adrY	& \adrY	& \adrY	& Событие №32.	\tabularnewline \hline
	732		& \adrY	& \adrY	& \adrY	& \adrY	& \adrY	& Событие №33.	\tabularnewline \hline
	733		& \adrY	& \adrY	& \adrY	& \adrY	& \adrY	& Событие №34.	\tabularnewline \hline
	734		& \adrY	& \adrY	& \adrY	& \adrY	& \adrY	& Событие №35.	\tabularnewline \hline
	735		& \adrY	& \adrY	& \adrY	& \adrY	& \adrY	& Событие №36.	\tabularnewline \hline
	736		& \adrY	& \adrY	& \adrY	& \adrY	& \adrY	& Событие №37.	\tabularnewline \hline
	737		& \adrY	& \adrY	& \adrY	& \adrY	& \adrY	& Событие №38.	\tabularnewline \hline
	738		& \adrY	& \adrY	& \adrY	& \adrY	& \adrY	& Событие №39.	\tabularnewline \hline
	739		& \adrY	& \adrY	& \adrY	& \adrY	& \adrY	& Событие №40.	\tabularnewline \hline
	
	\multicolumn{7}{|c|}{Команды передатчика поступившие с дискретных входов.} 		\tabularnewline \hline
	740		&		& \adrY	& \adrY	& \adrY	& \adrY	&Команда передатчика 1 (ДВ).	\tabularnewline \hline
	741		& 		& \adrY	& \adrY	& \adrY	& \adrY	&Команда передатчика 2 (ДВ).	\tabularnewline \hline
	742		& 		& \adrY	& \adrY	& \adrY	& \adrY	&Команда передатчика 3 (ДВ).	\tabularnewline \hline
	743		& 		& \adrY	& \adrY	& \adrY	& \adrY	&Команда передатчика 4 (ДВ).	\tabularnewline \hline
	744		&		& 		& \adrY	& \adrY	& \adrY	&Команда передатчика 5 (ДВ).	\tabularnewline \hline
	745		& 		& 		& \adrY	& \adrY	& \adrY	&Команда передатчика 6 (ДВ).	\tabularnewline \hline
	746		& 		& 		& \adrY	& \adrY	& \adrY	&Команда передатчика 7 (ДВ).	\tabularnewline \hline
	747		& 		&		& \adrY	& \adrY	& \adrY	&Команда передатчика 8 (ДВ).	\tabularnewline \hline
	747		& 		& 		& \adrY	& \adrY	& \adrY	&Команда передатчика 9 (ДВ).	\tabularnewline \hline
	749		& 		& 		& \adrY	& \adrY	& \adrY	&Команда передатчика 10 (ДВ).	\tabularnewline \hline
	750		& 		& 		& \adrY	& \adrY	& \adrY	&Команда передатчика 11 (ДВ).	\tabularnewline \hline
	751		& 		& 		& \adrY	& \adrY	& \adrY	&Команда передатчика 12 (ДВ).	\tabularnewline \hline
	752		& 		& 		& \adrY	& \adrY	& \adrY	&Команда передатчика 13 (ДВ).	\tabularnewline \hline
	753		& 		& 		& \adrY	& \adrY	& \adrY	&Команда передатчика 14 (ДВ).	\tabularnewline \hline
	754		& 		& 		& \adrY	& \adrY	& \adrY	&Команда передатчика 15 (ДВ).	\tabularnewline \hline
	755		& 		& 		& \adrY	& \adrY	& \adrY	&Команда передатчика 16 (ДВ).	\tabularnewline \hline
	756		&		& 		& \adrY	& \adrY	& \adrY	&Команда передатчика 17 (ДВ).	\tabularnewline \hline
	757		& 		& 		& \adrY	& \adrY	& \adrY	&Команда передатчика 18 (ДВ).	\tabularnewline \hline
	758		& 		& 		& \adrY	& \adrY	& \adrY	&Команда передатчика 19 (ДВ).	\tabularnewline \hline
	759		& 		& 		& \adrY	& \adrY	& \adrY	&Команда передатчика 20 (ДВ).	\tabularnewline \hline
	760		& 		& 		& \adrY	& \adrY	& \adrY	&Команда передатчика 21 (ДВ).	\tabularnewline \hline
	761		& 		& 		& \adrY	& \adrY	& \adrY	&Команда передатчика 22 (ДВ).	\tabularnewline \hline
	762		& 		& 		& \adrY	& \adrY	& \adrY	&Команда передатчика 23 (ДВ).	\tabularnewline \hline
	763		& 		& 		& \adrY	& \adrY	& \adrY	&Команда передатчика 24 (ДВ).	\tabularnewline \hline
	764		& 		& 		& \adrY	& \adrY	& \adrY	&Команда передатчика 25 (ДВ).	\tabularnewline \hline
	765		& 		& 		& \adrY	& \adrY	& \adrY	&Команда передатчика 26 (ДВ).	\tabularnewline \hline
	766		& 		&		& \adrY	& \adrY	& \adrY	&Команда передатчика 27 (ДВ).	\tabularnewline \hline
	767		& 		& 		& \adrY	& \adrY	& \adrY	&Команда передатчика 28 (ДВ).	\tabularnewline \hline
	768		& 		& 		& \adrY	& \adrY	& \adrY	&Команда передатчика 29 (ДВ).	\tabularnewline \hline
	769		& 		& 		& \adrY	& \adrY	& \adrY	&Команда передатчика 30 (ДВ).	\tabularnewline \hline
	770		& 		&		& \adrY	& \adrY	& \adrY	&Команда передатчика 31 (ДВ).	\tabularnewline \hline
	771		& 		&		& \adrY	& \adrY	& \adrY	&Команда передатчика 32 (ДВ).	\tabularnewline \hline
	
	\multicolumn{7}{|c|}{Команды передатчика поступившие с цифрового переприема.} 	\tabularnewline \hline
	772		& 		& \adrY	& \adrY	& \adrY	& \adrY	&Команда передатчика 1 (ЦПП).	\tabularnewline \hline
	773		& 		& \adrY	& \adrY	& \adrY	& \adrY	&Команда передатчика 2 (ЦПП).	\tabularnewline \hline
	774		& 		& \adrY	& \adrY	& \adrY	& \adrY	&Команда передатчика 3 (ЦПП).	\tabularnewline \hline
	775		& 		& \adrY	& \adrY	& \adrY	& \adrY	&Команда передатчика 4 (ЦПП).	\tabularnewline \hline
	776		& 		& 		& \adrY	& \adrY	& \adrY	&Команда передатчика 5 (ЦПП).	\tabularnewline \hline
	777		& 		& 		& \adrY	& \adrY	& \adrY	&Команда передатчика 6 (ЦПП).	\tabularnewline \hline
	778		& 		& 		& \adrY	& \adrY	& \adrY	&Команда передатчика 7 (ЦПП).	\tabularnewline \hline
	779		& 		& 		& \adrY	& \adrY	& \adrY	&Команда передатчика 8 (ЦПП).	\tabularnewline \hline
	780		& 		& 		& \adrY	& \adrY	& \adrY	&Команда передатчика 9 (ЦПП).	\tabularnewline \hline
	781		& 		& 		& \adrY	& \adrY	& \adrY	&Команда передатчика 10 (ЦПП).	\tabularnewline \hline
	782		& 		& 		& \adrY	& \adrY	& \adrY	&Команда передатчика 11 (ЦПП).	\tabularnewline \hline
	783		& 		& 		& \adrY	& \adrY	& \adrY	&Команда передатчика 12 (ЦПП).	\tabularnewline \hline
	784		& 		& 		& \adrY	& \adrY	& \adrY	&Команда передатчика 13 (ЦПП).	\tabularnewline \hline
	785		& 		& 		& \adrY	& \adrY	& \adrY	&Команда передатчика 14 (ЦПП).	\tabularnewline \hline
	786		& 		& 		& \adrY	& \adrY	& \adrY	&Команда передатчика 15 (ЦПП).	\tabularnewline \hline
	787		& 		& 		& \adrY	& \adrY	& \adrY	&Команда передатчика 16 (ЦПП).	\tabularnewline \hline
	788		& 		& 		& \adrY	& \adrY	& \adrY	&Команда передатчика 17 (ЦПП).	\tabularnewline \hline
	789		& 		& 		& \adrY	& \adrY	& \adrY	&Команда передатчика 18 (ЦПП).	\tabularnewline \hline
	790		& 		& 		& \adrY	& \adrY	& \adrY	&Команда передатчика 19 (ЦПП).	\tabularnewline \hline
	791		& 		& 		& \adrY	& \adrY	& \adrY	&Команда передатчика 20 (ЦПП).	\tabularnewline \hline
	792		& 		& 		& \adrY	& \adrY	& \adrY	&Команда передатчика 21 (ЦПП).	\tabularnewline \hline
	793		& 		& 		& \adrY	& \adrY	& \adrY	&Команда передатчика 22 (ЦПП).	\tabularnewline \hline
	794		& 		& 		& \adrY	& \adrY	& \adrY	&Команда передатчика 23 (ЦПП).	\tabularnewline \hline
	795		& 		& 		& \adrY	& \adrY	& \adrY	&Команда передатчика 24 (ЦПП).	\tabularnewline \hline
	796		& 		& 		& \adrY	& \adrY	& \adrY	&Команда передатчика 25 (ЦПП).	\tabularnewline \hline
	797		& 		& 		& \adrY	& \adrY	& \adrY	&Команда передатчика 26 (ЦПП).	\tabularnewline \hline
	798		& 		& 		& \adrY	& \adrY	& \adrY	&Команда передатчика 27 (ЦПП).	\tabularnewline \hline
	799		& 		& 		& \adrY	& \adrY	& \adrY	&Команда передатчика 28 (ЦПП).	\tabularnewline \hline
	800		& 		& 		& \adrY	& \adrY	& \adrY	&Команда передатчика 29 (ЦПП).	\tabularnewline \hline
	801		& 		& 		& \adrY	& \adrY	& \adrY	&Команда передатчика 30 (ЦПП).	\tabularnewline \hline
	802		& 		& 		& \adrY	& \adrY	& \adrY	&Команда передатчика 31 (ЦПП).	\tabularnewline \hline
	803		& 		& 		& \adrY	& \adrY	& \adrY	&Команда передатчика 32 (ЦПП).	\tabularnewline \hline
	
	\multicolumn{7}{|c|}{Команды приемника (первого).} 								\tabularnewline \hline
	804		& 		& \adrY	& \adrY	& \adrY	& \adrY	& Команда приемника 1.			\tabularnewline \hline
	805		& 		& \adrY	& \adrY	& \adrY	& \adrY	& Команда приемника 2.			\tabularnewline \hline
	806		& 		& \adrY	& \adrY	& \adrY	& \adrY	& Команда приемника 3.			\tabularnewline \hline
	807		& 		& \adrY	& \adrY	& \adrY	& \adrY	& Команда приемника 4.			\tabularnewline \hline
	808		& 		& 		& \adrY	& \adrY	& \adrY	& Команда приемника 5.			\tabularnewline \hline
	809		& 		& 		& \adrY	& \adrY	& \adrY	& Команда приемника 6.			\tabularnewline \hline
	810		& 		& 		& \adrY	& \adrY	& \adrY	& Команда приемника 7.			\tabularnewline \hline
	811		& 		& 		& \adrY	& \adrY	& \adrY	& Команда приемника 8.			\tabularnewline \hline
	812		& 		& 		& \adrY	& \adrY	& \adrY	& Команда приемника 9.			\tabularnewline \hline
	813		& 		& 		& \adrY	& \adrY	& \adrY	& Команда приемника 10.			\tabularnewline \hline
	814		& 		& 		& \adrY	& \adrY	& \adrY	& Команда приемника 11.			\tabularnewline \hline
	815		& 		& 		& \adrY	& \adrY	& \adrY	& Команда приемника 12.			\tabularnewline \hline
	816		& 		& 		& \adrY	& \adrY	& \adrY	& Команда приемника 13.			\tabularnewline \hline
	817		& 		& 		& \adrY	& \adrY	& \adrY	& Команда приемника 14.			\tabularnewline \hline
	818		& 		& 		& \adrY	& \adrY	& \adrY	& Команда приемника 15.			\tabularnewline \hline
	819		& 		& 		& \adrY	& \adrY	& \adrY	& Команда приемника 16.			\tabularnewline \hline
	820		& 		& 		& \adrY	& \adrY	& \adrY	& Команда приемника 17.			\tabularnewline \hline
	821		& 		& 		& \adrY	& \adrY	& \adrY	& Команда приемника 18.			\tabularnewline \hline
	822		& 		& 		& \adrY	& \adrY	& \adrY	& Команда приемника 19.			\tabularnewline \hline
	823		& 		& 		& \adrY	& \adrY	& \adrY	& Команда приемника 20.			\tabularnewline \hline
	824		& 		& 		& \adrY	& \adrY	& \adrY	& Команда приемника 21.			\tabularnewline \hline
	825		& 		& 		& \adrY	& \adrY	& \adrY	& Команда приемника 22.			\tabularnewline \hline
	826		& 		& 		& \adrY	& \adrY	& \adrY	& Команда приемника 23.			\tabularnewline \hline
	827		& 		& 		& \adrY	& \adrY	& \adrY	& Команда приемника 24.			\tabularnewline \hline
	828		& 		& 		& \adrY	& \adrY	& \adrY	& Команда приемника 25.			\tabularnewline \hline
	829		& 		& 		& \adrY	& \adrY	& \adrY	& Команда приемника 26.			\tabularnewline \hline
	830		& 		& 		& \adrY	& \adrY	& \adrY	& Команда приемника 27.			\tabularnewline \hline
	831		& 		& 		& \adrY	& \adrY	& \adrY	& Команда приемника 28.			\tabularnewline \hline
	932		& 		& 		& \adrY	& \adrY	& \adrY	& Команда приемника 29.			\tabularnewline \hline
	933		& 		& 		& \adrY	& \adrY	& \adrY	& Команда приемника 30.			\tabularnewline \hline
	934		& 		& 		& \adrY	& \adrY	& \adrY	& Команда приемника 31.			\tabularnewline \hline
	935		& 		& 		& \adrY	& \adrY	& \adrY	& Команда приемника 32.			\tabularnewline \hline
	 
	\lasthline
\end{tabularx}






\end{document}

