%%% ----------
\section{Технические характеристики} \label{sec:tth}

\begin{list}{--}{Технические характеристики:}
	\item Физический уровень: последовательный канал стандарта RS-485/RS-422. Цепи подключения линии гальванически развязаны с~остальными цепями и~корпусом аппарата (уровень изоляции 2500~эфф.).
	\item Максимальная длина линии <<пункт управления - <<АВАНТ>>>> определеятеся типом кабеля и~скоростью передачи. Рекомендуется использовать экранированную витую пару.
	\item Скорости обмена данными: 600, 1200, 2400, 4800, 9600, 19200~бит/с.
	\item Адреса устройства в~сети МЭК 870-5-101: от~1 до~247.
\end{list}

При передаче данных исползуется формат кадра FT1.2, определенный в ГОСТ~Р~МЭК~870-5-2. Допускается формат как с~фиксированной, так и с~переменной длиной блока.

Аппаратура <<АВАНТ>> поддерживает только небалансную передачу по каналу.

Для передачи данных используется только режим <<1>> (младший байт передается первым). 
 
В~случае обнаружения ошибки прием кадра прекращается. Для возобновления процесса приема кадров требуется, чтобы приемник зафиксировал, что входная линия перешла в~исходное состояние (уровень сигнала в~линии соответсвует уровню стопового бита) и~удерживается в~этом состоянии в~течение времени передачи не менее 33~бит.

\begin{list}{--}{Поддерживаются следующие стандартные идентификаторы типа (ASDU):}
	\item M\_SP\_NA\_1 1 Одноэлементная информация без~метки времени.
	\item M\_SP\_TB\_1 30 Одноэлементная информация с~меткой времени \\ СР56Время2а.
	\item M\_EI\_NA\_1 70 Окончение инициализации.
	\item C\_IC\_NA\_1 100 Команда опроса.
	\item C\_CS\_NA\_1 103 Команда синхронизации часов.
\end{list}

\begin{list}{--}{Фиксированные параметры пакета МЭК~870-5-101:} 
	\item Общий адрес ASDU : 1.
	\item Длина адреса станции (байт): 1.
	\item Длина общего адреса ASDU (байт): 1.
	\item Длина адреса объекта информации (байт): 2.
	\item Длина кода причины передачи (байт): 1.
\end{list}

Процедура опроса обеспечивается на~канальном уровне, который запрашивает пользовательские данные классов~1 и 2. 

Данные класса~2 можно считать командой опроса, при этом они будут переданы без метки времени.